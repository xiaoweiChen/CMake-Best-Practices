
\documentclass[11pt,a4paper,UTF8]{book}

\usepackage[T1]{fontenc}
\usepackage[utf8]{inputenc}
\usepackage{authblk}

\usepackage{fontspec}                  %引入字体设置宏包
\setmainfont{Times New Roman}             %设置英文正文字体
% Courier New
% Book Antique
\setsansfont{Arial}                    %英文无衬线字体
\setmonofont{Courier New}              %英文等宽字体

\usepackage{ctex} %导入中文包
%\usepackage{ulem}
\usepackage{tocvsec2}
\usepackage{verbatim}

\usepackage{tabularx}
\usepackage{booktabs} 
\usepackage{multirow}
\usepackage{bbding}
\usepackage{float}
\usepackage{xspace}
\usepackage[none]{hyphenat}

\usepackage{graphicx}
\usepackage{subfigure}
\usepackage{pifont}

\usepackage{hyperref}  %制作pdf的目录
\usepackage{subfiles} %使用多文件方式进行

\usepackage{geometry} %设置页边距的包
\geometry{left=2.5cm,right=2cm,top=2.54cm,bottom=2.54cm} %设置书籍的页边距

\usepackage{url}
\hypersetup{hidelinks, %去红框
	colorlinks=true,
	allcolors=black,
	pdfstartview=Fit,
	breaklinks=true
}

% 调整itemlist中的行间距
\usepackage{enumitem}
\setenumerate[1]{itemsep=0pt,partopsep=0pt,parsep=\parskip,topsep=5pt}
\setitemize[1]{itemsep=0pt,partopsep=0pt,parsep=\parskip,topsep=5pt}
\setdescription{itemsep=0pt,partopsep=0pt,parsep=\parskip,topsep=5pt}

% 超链接样式设置
\usepackage{hyperref}
\hypersetup{
	colorlinks=true,
	linkcolor=blue,
	filecolor=blue,
	urlcolor=blue,
	citecolor=cyan,
}

\usepackage{indentfirst}

\usepackage{listings}
\usepackage[usenames,dvipsnames,svgnames, x11names]{xcolor}

\usepackage[most]{tcolorbox}

%展示代码
\definecolor{mygreen}{rgb}{0,0.6,0}
\definecolor{mygray}{rgb}{0.5,0.5,0.5}
\definecolor{mymauve}{rgb}{0.58,0,0.82}
\definecolor{keywordcolor}{rgb}{0.8,0.1,0.5}
\definecolor{webgreen}{rgb}{0,.5,0}
\definecolor{bgcolor}{rgb}{0.92,0.92,0.92}

%定义CMake
\lstdefinelanguage{CMake}
{morekeywords={
		cmake\_minimum\_required,
		project,
		add\_executable,
		add\_library,
		target\_link\_libraries,
		cmake\_parse\_arguments,
		cmake\_language,
		set, unset,
		option,
		string,
		list,
		math,
		message,
		if, elseif, else, endif,
		mark\_as\_advanced,
		foreach, endforeach,
		while, endwhile,
		add\_subdirectory, include, return, include\_gurad,
		function, endfunction,
		macro, endmacro,
		find\_package,
		cmake\_push\_check\_state,
		cmake\_pop\_check\_state,
		cmake\_reset\_check\_state,
		add\_test,
		set\_tests\_properties, 
		check\_c\_source\_runs,
		check\_cxx\_source\_runs,
		check\_fortran\_source\_runs,
		check\_source\_runs,
		check\_compiler\_flag,
		check\_c\_compiler\_flag,
		check\_cxx\_compiler\_flag,
		check\_fortran\_compiler\_flag,
		check\_symbol\_exists,
		check\_cxx\_symbol\_exists,
		check\_linker\_flag,
		cmake\_policy,
		set\_property,
		get\_property,
		define\_property,
		get\_cmake\_property,
		set\_cmake\_property,
		set\_target\_properties,
		get\_target\_property,
		set\_directory\_properties,
		get\_directory\_property,
		set\_source\_files\_properties,
		get\_source\_file\_property,
		set\_tests\_properties,
		get\_tests\_property,
		get\_test\_property,
		cmake\_print\_properties,
		cmake\_print\_variables,
		variable\_watch,
		include\_guard,
		target\_link\_options,
		target\_compile\_definitions,
		target\_compile\_options,
		include\_directories,
		add\_definitions,
		remove\_definitions,
		add\_compile\_definitions,
		add\_compile\_options,
		link\_libraries,
		link\_directories,
		add\_link\_options,
		target\_include\_directories,
		target\_compile\_features,
		add\_custom\_command,
		add\_custom\_target,
		execute\_process,
		cmake\_path,
		get\_filename\_component,
		file,
		configure\_file,
		generate\_export\_header,
		export,
		find\_file,
		find\_library,
		find\_package,
		find\_program,
		pkg\_check\_modules,
		pkg\_search\_module,
		pkg\_get\_variable,
		add\_test,
		enable\_testing,
		set\_tests\_properties,
		site\_name,
		ctest\_empty\_binary\_directory,
		ctest\_start,
		ctest\_configure,
		ctest\_submit,
		ctest\_build,
		ctest\_memcheck,
		ctest\_upload,
		ctest\_test,
		gtest\_add\_tests,
		gtest\_discover\_tests,
		install,
		write\_basic\_package\_version\_file,
		configure\_package\_config\_file,
		cpack\_add\_component,
		cpack\_add\_install\_type,
		cpack\_add\_component\_group,
		ExternalProject\_Add,
		ExternalProject\_Add\_StepDependencies,
		ExternalProject\_Get\_Property,
		ExternalProject\_Add\_Step,
		FetchContent\_Declare,
		FetchContent\_GetProperties,
		FetchContent\_Populate,
		source\_group,
		target\_precompile\_headers,
		qt5\_wrap\_cpp,
		qt5\_wrap\_ui,
		qt5\_add\_resources,
		qt5\_add\_big\_resources,
		qt5\_add\_binary\_resources,
		qt5\_add\_translation,
		qt5\_create\_translation,
		compile\_definitions,
		add\_llvm\_component\_library,
		add\_llvm\_tool,
		llvm\_multisource,
		llvm\_test\_data,
		doxygen\_add\_docs,
		cmake\_dependent\_option,
		target\_sources,
		conan\_cmake\_autodetect,
		conan\_cmake\_configure,
		conan\_cmake\_install,
		doxygen\_add\_docs,
		check\_source\_compiles,
		check\_language,
		enable\_language,
		add\_dependencies,
		find\_path,
		find\_package\_handle\_standard\_args,
	}, %定义关键字
	sensitive=false, %是否大小写敏感
	morecomment=[l]{\#},
	morestring=[b]",
	morestring=[d]',
}

\lstdefinestyle{styleCXX}{
	language = C++,  
	backgroundcolor=\color{blue!3!white}, 
	%basicstyle = \footnotesize,  
	basicstyle      =   \zihao{-5}\ttfamily,
	numberstyle     =   \zihao{-5}\ttfamily,   
	%breakatwhitespace = false,    
	basewidth       =   0.5em,    
	breaklines = true,                 
	captionpos = b,                    
	commentstyle = \color{mygray}\bfseries,
	%extendedchars = false,             
	frame =shadowbox, 
	framerule=0.5pt,
	%frameround = fttt,
	keepspaces=true,
	keywordstyle=\color{blue}\bfseries, % keyword style
	otherkeywords={string}, 
	numbers=left, 
	numbersep=5pt,
	numberstyle=\tiny\color{mygray},
	rulecolor=\color{black},         
	%showspaces=false,  
	%showstringspaces=false, 
	%showtabs=false,    
	%stepnumber=1,         
	stringstyle=\color{mymauve},        % string literal style
	tabsize=2,          
	columns         =   fixed,
	flexiblecolumns,                   
}


\lstdefinestyle{styleCMake}{
	language=CMake,
	backgroundcolor=\color{blue!3!white}, 
	basicstyle=\tt, 
	breakatwhitespace = false,
	breaklines = true,
	captionpos = b,
	commentstyle = \color{mygray}\bfseries, 
	extendedchars =false,             
	frame=shadowbox, 
	tabsize=2,
	framerule=0.5pt,
	keepspaces=true,
	keywordstyle=\color{blue}\bfseries, % keyword style
	otherkeywords={string}, 
	rulecolor=\color{black},
	showspaces=false,
	showstringspaces=false,
	showtabs=false,
	stepnumber=1,
	stringstyle=\color{purple},        % string literal style
}

\lstdefinestyle{stylePython}{
	language        =   Python, % 语言选Python
	backgroundcolor=\color{blue!3!white}, 
	basicstyle      =   \zihao{-5}\ttfamily,
	numberstyle     =   \zihao{-5}\ttfamily,
	keywordstyle    =   \color{blue},
	keywordstyle    =   [2] \color{teal},
	stringstyle     =   \color{magenta},
	commentstyle    =   \color{red}\ttfamily,
	frame = shadowbox, 
	breaklines      =   true,   % 自动换行,建议不要写太长的行
	columns         =   fixed,  % 如果不加这一句,字间距就不固定,很丑,必须加
	basewidth       =   0.5em,
	%basicstyle          =   \sffamily,          % 基本代码风格
	%keywordstyle        =   \bfseries,          % 关键字风格
	%commentstyle        =   \rmfamily\itshape,  % 注释的风格,斜体
	%stringstyle         =   \ttfamily,  % 字符串风格
	flexiblecolumns,                % 别问为什么,加上这个
	%numbers             =   left,   % 行号的位置在左边
	showspaces          =   false,  % 是否显示空格,显示了有点乱,所以不现实了
	numberstyle         =   \zihao{-5}\ttfamily,    % 行号的样式,小五号,tt等宽字体
	showstringspaces    =   false,
	captionpos          =   t,      % 这段代码的名字所呈现的位置,t指的是top上面
	frame               =   lrtb,   % 显示边框
	tabsize=2,  
}

\tcbset{
	commandshell/.style={
		listing only,
		colback=black!75!white,
		colupper=white,
		lowerbox=ignored,
		listing options={
			language={bash},
			basicstyle=\ttfamily,
			columns = fixed,
			flexiblecolumns
		}
}}

\usepackage{tikz}

% URL 正确换行
% https://liam.page/2017/05/17/help-the-url-command-from-hyperref-to-break-at-line-wrapping-point/
\makeatletter
\def\UrlAlphabet{%
	\do\a\do\b\do\c\do\d\do\e\do\f\do\g\do\h\do\i\do\j%
	\do\k\do\l\do\m\do\n\do\o\do\p\do\q\do\r\do\s\do\t%
	\do\u\do\v\do\w\do\x\do\y\do\z\do\A\do\B\do\C\do\D%
	\do\E\do\F\do\G\do\H\do\I\do\J\do\K\do\L\do\M\do\N%
	\do\O\do\P\do\Q\do\R\do\S\do\T\do\U\do\V\do\W\do\X%
	\do\Y\do\Z}
\def\UrlDigits{\do\1\do\2\do\3\do\4\do\5\do\6\do\7\do\8\do\9\do\0}
\g@addto@macro{\UrlBreaks}{\UrlOrds}
\g@addto@macro{\UrlBreaks}{\UrlAlphabet}
\g@addto@macro{\UrlBreaks}{\UrlDigits}
\makeatother

% enable subsubsubsection
% from https://tex.stackexchange.com/练习题/274212/correct-hierarchy-levels-of-pdf-bookmarks-for-custom-section-subsubsubsection
\usepackage[depth=3]{bookmark}
\setcounter{secnumdepth}{3}
\setcounter{tocdepth}{4}
\hypersetup{bookmarksdepth=4}

\makeatletter

\newcommand{\toclevel@subsubsubsection}{4}
\newcounter{subsubsubsection}[subsubsection]

\renewcommand{\thesubsubsubsection}{\thesubsubsection.\arabic{subsubsubsection}}

\newcommand{\subsubsubsection}{\@startsection{subsubsubsection}{4}{\z@}%
	{-3.25ex\@plus -1ex \@minus -.2ex}%
	{1.5ex \@plus .2ex}%
	{\normalfont\normalsize\bf\bfseries}}

\newcommand*{\l@subsubsubsection}{\@dottedtocline{4}{11em}{5em}}  

\newcommand{\subsubsubsectionmark}[1]{}
\makeatother

\begin{document}
\begin{sloppypar} %latex中一行文字出现溢出问题的解决方法
	%\maketitle
	
	\begin{center}
		\thispagestyle{empty}
		%\includegraphics[width=\textwidth,height=\textheight,keepaspectratio]{cover.jpg}
		\begin{tikzpicture}[remember picture, overlay, inner sep=0pt]
			\node at (current page.center) 
			{\includegraphics[width=\paperwidth, keepaspectratio=false]{cover.jpg}};
		\end{tikzpicture}
		\newpage
		\thispagestyle{empty}
		\huge
		\textbf{CMake Best Practices} 
		\\[9pt]
		\normalsize
		Discover proven techniques for creating and maintaining programming projects with CMake
		\\[9pt]
		\normalsize
		(使用CMake创建和维护项目)
		\\[10pt]
		\normalsize 
		作者: Dominik Berner, Mustafa Kemal Gilor  
		\\[8pt]
		\normalsize
		译者:陈晓伟
	\end{center}
	
	\hspace*{\fill} \\ %插入空行
	\noindent\textbf{本书概述}
	
	CMake是一个强大的工具,用于执行各种各样的任务,需要一个起点对CMake进行学习。本书更专注于常见的任务,边实践边学习CMake。CMake文档很全面,但缺少具有代表性的例子,说明如何将源码组合在一起,特别是互联网上还有有很奇淫技巧和过时的解决方案。本书的重点是帮助读者把需要做的事情串在一起,编写CMake,从而创建简洁和可维护的项目。
	
	阅读本书后,不仅可以掌握CMake的基础知识,还可以通过构建大型复杂项目和在任何编程环境中运行的构建示例。还可以使用集成和自动化工具,以提高整体软件的质量,例如:测试框架、模糊测试和自动生成文档。编写代码只是工作的一半,本书还会引导读者对程序进行安装、打包和分发,这些都是为使用现代化的CI/CD工作流程需要的基础功能。
	
	阅读完本书,将能以最好的方式使用CMake建立和维护复杂的软件项目。
	
	
	\hspace*{\fill} \\ %插入空行
	\noindent\textbf{关键特性}
	\begin{itemize}
		\item 理解CMake是什么,如何工作和交互
		\item 了解如何正确地创建和维护结构CMake项目
		\item 探索工具和技术,以最大程度的使用CMake管理项目
	\end{itemize}

	\hspace*{\fill} \\ %插入空行
	\noindent\textbf{将会学到}
	\begin{itemize}
		\item 构建结构良好的CMake项目
		\item 跨项目模块化和重用CMake代码
		\item 将各种静态分析、检测、格式化和文档生成工具集成到CMake项目中
		\item 尝试进行跨平台构建
		\item 了解如何使用不同的工具链
		\item 为项目构建一个定义良好,且可移植的构建环境
	\end{itemize}
	
	\hspace*{\fill} \\ %插入空行
	\noindent\textbf{作者简介}
	
	\textbf{Dominik Berner}是一位拥有20年专业软件开发经验的软件工程师、博客作者和演讲者。主要使用C++,还参与了许多软件项目,从为初创公司的外科手术模拟器编写软件,到为MedTech行业的大型企业维护大型平台,再到为介于两者之间的公司创建物联网解决方案。他相信,良好的设计和可维护的构建环境是使团队高效编写软件,并创建高质量软件的关键因素之一。当他不写代码时,会去博客写一些文章,或者在会议上发表一些关于软件开发的演讲。
	
	\textbf{Mustafa Kemal Gilor}是一位经验丰富的专业人员,从事电信、国防工业和开源软件的性能关键软件开发。他的专长是高性能和可扩展的软件设计、网络技术、DevOps和软件架构。他对计算机的兴趣在童年时期就显现出来了。他在12岁左右学习编程破解MMORPG游戏,从那时起他就一直在编写软件。他最喜欢的编程语言是C++,并且喜欢做框架设计和系统编程,也是CMake的坚定倡导者。他的职业生涯中,维护了许多代码库,并将许多遗留(非CMake)项目移植到CMake。
	
	\hspace*{\fill} \\ %插入空行
	\noindent\textbf{审评者介绍}
	
	\textbf{Richard Von Lehe}住在明尼苏达州的双城地区。过去的几年中,他在软件项目中经常使用CMake,包括正畸建模、建筑控制、无人机避免碰撞和专用打印机。闲暇时,他喜欢和家人,以及宠物兔Gus一起放松,还喜欢骑自行车和弹吉他。
	
	\textbf{Toni Solarin-Solada}是一名软件工程师,专门设计抽象底层操作系统服务的跨平台编程库。
	
	
	\hspace*{\fill} \\ %插入空行
	\noindent\textbf{本书相关}
	\begin{itemize}
	\item Github地址:\url{https://github.com/xiaoweiChen/CMake-Best-Practices}
	\end{itemize}
	\newpage
	
	%前言
	\pagestyle{empty}
	\subfile{content/preface.tex}
	\newpage
	
	\tableofcontents
	\newpage

	\setsecnumdepth{section}
	
	\color{white}
	\section*{\zihao{1}第一部分:基础知识}
	\pagecolor{mygray}
	\addcontentsline{toc}{section}{第一部分:基础知识}
	\textbf{\subfile{content/1/Section.tex}}
	\newpage
	\color{black}
	\pagecolor{white}

	\subsection*{\zihao{2} 第1章\hspace{0.5cm}启用CMake}
	\addcontentsline{toc}{subsection}{第1章\hspace{0.5cm}启用CMake}
	\subfile{content/1/chapter1/0.tex}
	
	\subsubsection*{\zihao{3} 1.1.\hspace{0.2cm}相关准备}
	\addcontentsline{toc}{subsubsection}{1.1.\hspace{0.2cm}相关准备}
	\subfile{content/1/chapter1/1.tex}
	
	\subsubsection*{\zihao{3} 1.2.\hspace{0.2cm}简介CMake}
	\addcontentsline{toc}{subsubsection}{1.2.\hspace{0.2cm}简介CMake}
	\subfile{content/1/chapter1/2.tex}
	
	\subsubsection*{\zihao{3} 1.3.\hspace{0.2cm}安装CMake}
	\addcontentsline{toc}{subsubsection}{1.3.\hspace{0.2cm}安装CMake}
	\subfile{content/1/chapter1/3.tex}
	
	\subsubsection*{\zihao{3} 1.4.\hspace{0.2cm}构建第一个工程}
	\addcontentsline{toc}{subsubsection}{1.4.\hspace{0.2cm}构建第一个工程}
	\subfile{content/1/chapter1/4.tex}
	
	\subsubsection*{\zihao{3} 1.5.\hspace{0.2cm}了解CMake的构建过程}
	\addcontentsline{toc}{subsubsection}{1.5.\hspace{0.2cm}了解CMake的构建过程}
	\subfile{content/1/chapter1/5.tex}
	
	\subsubsection*{\zihao{3} 1.6.\hspace{0.2cm}书写CMake文件}
	\addcontentsline{toc}{subsubsection}{1.6.\hspace{0.2cm}书写CMake文件}
	\subfile{content/1/chapter1/6.tex}
	
	\subsubsection*{\zihao{3} 1.7.\hspace{0.2cm}不同工具链和构建配置}
	\addcontentsline{toc}{subsubsection}{1.7.\hspace{0.2cm}不同工具链和构建配置}
	\subfile{content/1/chapter1/7.tex}
	
	\subsubsection*{\zihao{3} 1.8.\hspace{0.2cm}使用预设置维护良好的构建配置}
	\addcontentsline{toc}{subsubsection}{1.8.\hspace{0.2cm}使用预设置维护良好的构建配置}
	\subfile{content/1/chapter1/8.tex}
	
	\subsubsection*{\zihao{3} 1.9.\hspace{0.2cm}总结}
	\addcontentsline{toc}{subsubsection}{1.9.\hspace{0.2cm}总结}
	\subfile{content/1/chapter1/9.tex}
	
	\subsubsection*{\zihao{3} 1.10.\hspace{0.2cm}扩展阅读}
	\addcontentsline{toc}{subsubsection}{1.10.\hspace{0.2cm}扩展阅读}
	\subfile{content/1/chapter1/10.tex}
	
	\subsubsection*{\zihao{3} 1.11.\hspace{0.2cm}练习题}
	\addcontentsline{toc}{subsubsection}{1.11.\hspace{0.2cm}练习题}
	\subfile{content/1/chapter1/11.tex}
	\newpage
	
	\subsection*{\zihao{2} 第2章\hspace{0.5cm}CMake的最佳使用方式}
	\addcontentsline{toc}{subsection}{第2章\hspace{0.5cm}CMake的最佳使用方式}
	\subfile{content/1/chapter2/0.tex}
	
	\subsubsection*{\zihao{3} 2.1.\hspace{0.2cm}相关准备}
	\addcontentsline{toc}{subsubsection}{2.1.\hspace{0.2cm}相关准备}
	\subfile{content/1/chapter2/1.tex}
	
	\subsubsection*{\zihao{3} 2.2.\hspace{0.2cm}通过命令行使用CMake}
	\addcontentsline{toc}{subsubsection}{2.2.\hspace{0.2cm}通过命令行使用CMake}
	\subfile{content/1/chapter2/2.tex}
	
	\subsubsection*{\zihao{3} 2.3.\hspace{0.2cm}CMake-GUI和ccmake的高级配置}
	\addcontentsline{toc}{subsubsection}{2.3.\hspace{0.2cm}CMake-GUI和ccmake的高级配置}
	\subfile{content/1/chapter2/3.tex}
	
	\subsubsection*{\zihao{3} 2.4.\hspace{0.2cm}Visual Studio、Visual Studio Code和Qt Creator中使用CMake}
	\addcontentsline{toc}{subsubsection}{2.4.\hspace{0.2cm}Visual Studio、Visual Studio Code和Qt Creator中使用CMake}
	\subfile{content/1/chapter2/4.tex}
	
	\subsubsection*{\zihao{3} 2.5.\hspace{0.2cm}总结}
	\addcontentsline{toc}{subsubsection}{2.5.\hspace{0.2cm}总结}
	\subfile{content/1/chapter2/5.tex}
	
	\subsubsection*{\zihao{3} 2.6.\hspace{0.2cm}练习题}
	\addcontentsline{toc}{subsubsection}{2.6.\hspace{0.2cm}练习题}
	\subfile{content/1/chapter2/6.tex}
	
	\subsubsection*{\zihao{3} 2.7.\hspace{0.2cm}扩展阅读}
	\addcontentsline{toc}{subsubsection}{2.7.\hspace{0.2cm}扩展阅读}
	\subfile{content/1/chapter2/7.tex}
	\newpage
	
	\subsection*{\zihao{2} 第3章\hspace{0.5cm}创建CMake项目}
	\addcontentsline{toc}{subsection}{第3章\hspace{0.5cm}创建CMake项目}
	\subfile{content/1/chapter3/0.tex}
	
	\subsubsection*{\zihao{3} 3.1.\hspace{0.2cm}相关准备}
	\addcontentsline{toc}{subsubsection}{3.1.\hspace{0.2cm}相关准备}
	\subfile{content/1/chapter3/1.tex}
	
	\subsubsection*{\zihao{3} 3.2.\hspace{0.2cm}创建项目}
	\addcontentsline{toc}{subsubsection}{3.2.\hspace{0.2cm}创建项目}
	\subfile{content/1/chapter3/2.tex}
	
	\subsubsection*{\zihao{3} 3.3.\hspace{0.2cm}创建“hello world”可执行文件}
	\addcontentsline{toc}{subsubsection}{3.3.\hspace{0.2cm}创建“hello world”可执行文件}
	\subfile{content/1/chapter3/3.tex}
	
	\subsubsection*{\zihao{3} 3.4.\hspace{0.2cm}创建库}
	\addcontentsline{toc}{subsubsection}{3.4.\hspace{0.2cm}创建库}
	\subfile{content/1/chapter3/4.tex}
	
	\subsubsection*{\zihao{3} 3.5.\hspace{0.2cm}将它们结合在一起——使用库}
	\addcontentsline{toc}{subsubsection}{3.5.\hspace{0.2cm}将它们结合在一起——使用库}
	\subfile{content/1/chapter3/5.tex}
	
	\subsubsection*{\zihao{3} 3.6.\hspace{0.2cm}总结}
	\addcontentsline{toc}{subsubsection}{3.6.\hspace{0.2cm}总结}
	\subfile{content/1/chapter3/6.tex}
	
	\subsubsection*{\zihao{3} 3.7.\hspace{0.2cm}练习题}
	\addcontentsline{toc}{subsubsection}{3.7.\hspace{0.2cm}练习题}
	\subfile{content/1/chapter3/7.tex}
	\newpage
	
	\color{white}
	\section*{\zihao{1}第二部分:实战CMake}
	\pagecolor{mygray}
	\addcontentsline{toc}{section}{第二部分:实战CMake}
	\textbf{\subfile{content/2/Section.tex}}
	\newpage
	\color{black}
	\pagecolor{white}
	
	\subsection*{\zihao{2} 第4章\hspace{0.5cm}打包、部署和安装}
	\addcontentsline{toc}{subsection}{第4章\hspace{0.5cm}打包、部署和安装}
	\subfile{content/2/chapter4/0.tex}
	
	\subsubsection*{\zihao{3} 4.1.\hspace{0.2cm}相关准备}
	\addcontentsline{toc}{subsubsection}{4.1.\hspace{0.2cm}相关准备}
	\subfile{content/2/chapter4/1.tex}
	
	\subsubsection*{\zihao{3} 4.2.\hspace{0.2cm}使目标可安装}
	\addcontentsline{toc}{subsubsection}{4.2.\hspace{0.2cm}使目标可安装}
	\subfile{content/2/chapter4/2.tex}
	
	\subsubsection*{\zihao{3} 4.3.\hspace{0.2cm}提供项目配置信息}
	\addcontentsline{toc}{subsubsection}{4.3.\hspace{0.2cm}提供项目配置信息}
	\subfile{content/2/chapter4/3.tex}
	
	\subsubsection*{\zihao{3} 4.4.\hspace{0.2cm}创建安装包}
	\addcontentsline{toc}{subsubsection}{4.4.\hspace{0.2cm}创建安装包}
	\subfile{content/2/chapter4/4.tex}
	
	\subsubsection*{\zihao{3} 4.5.\hspace{0.2cm}总结}
	\addcontentsline{toc}{subsubsection}{4.5.\hspace{0.2cm}总结}
	\subfile{content/2/chapter4/5.tex}
	
	\subsubsection*{\zihao{3} 4.6.\hspace{0.2cm}练习题}
	\addcontentsline{toc}{subsubsection}{4.6.\hspace{0.2cm}练习题}
	\subfile{content/2/chapter4/6.tex}
	\newpage
	
	\subsection*{\zihao{2} 第5章\hspace{0.5cm}集成第三方库和依赖管理}
	\addcontentsline{toc}{subsection}{第5章\hspace{0.5cm}集成第三方库和依赖管理}
	\subfile{content/2/chapter5/0.tex}
	
	\subsubsection*{\zihao{3} 5.1.\hspace{0.2cm}相关准备}
	\addcontentsline{toc}{subsubsection}{5.1.\hspace{0.2cm}相关准备}
	\subfile{content/2/chapter5/1.tex}
	
	\subsubsection*{\zihao{3} 5.2.\hspace{0.2cm}查找文件、程序和路径}
	\addcontentsline{toc}{subsubsection}{5.2.\hspace{0.2cm}查找文件、程序和路径}
	\subfile{content/2/chapter5/2.tex}
	
	\subsubsection*{\zihao{3} 5.3.\hspace{0.2cm}使用第三方库}
	\addcontentsline{toc}{subsubsection}{5.3.\hspace{0.2cm}使用第三方库}
	\subfile{content/2/chapter5/3.tex}
	
	\subsubsection*{\zihao{3} 5.4.\hspace{0.2cm}包管理器}
	\addcontentsline{toc}{subsubsection}{5.4.\hspace{0.2cm}包管理器}
	\subfile{content/2/chapter5/4.tex}
	
	\subsubsection*{\zihao{3} 5.5.\hspace{0.2cm}获取依赖项源代码}
	\addcontentsline{toc}{subsubsection}{5.5.\hspace{0.2cm}获取依赖项源代码}
	\subfile{content/2/chapter5/5.tex}
	
	\subsubsection*{\zihao{3} 5.6.\hspace{0.2cm}总结}
	\addcontentsline{toc}{subsubsection}{5.6.\hspace{0.2cm}总结}
	\subfile{content/2/chapter5/6.tex}
	
	\subsubsection*{\zihao{3} 5.7.\hspace{0.2cm}练习题}
	\addcontentsline{toc}{subsubsection}{5.7.\hspace{0.2cm}练习题}
	\subfile{content/2/chapter5/7.tex}
	\newpage
	
	\subsection*{\zihao{2} 第6章\hspace{0.5cm}自动生成文档}
	\addcontentsline{toc}{subsection}{第6章\hspace{0.5cm}自动生成文档}
	\subfile{content/2/chapter6/0.tex}
	
	\subsubsection*{\zihao{3} 6.1.\hspace{0.2cm}相关准备}
	\addcontentsline{toc}{subsubsection}{6.1.\hspace{0.2cm}相关准备}
	\subfile{content/2/chapter6/1.tex}
	
	\subsubsection*{\zihao{3} 6.2.\hspace{0.2cm}用代码生成文档}
	\addcontentsline{toc}{subsubsection}{6.2.\hspace{0.2cm}用代码生成文档}
	\subfile{content/2/chapter6/2.tex}
	
	\subsubsection*{\zihao{3} 6.3.\hspace{0.2cm}用CPack打包和分发文档}
	\addcontentsline{toc}{subsubsection}{6.3.\hspace{0.2cm}用CPack打包和分发文档}
	\subfile{content/2/chapter6/3.tex}
	
	\subsubsection*{\zihao{3} 6.4.\hspace{0.2cm}创建CMake目标的依赖关系图}
	\addcontentsline{toc}{subsubsection}{6.4.\hspace{0.2cm}创建CMake目标的依赖关系图}
	\subfile{content/2/chapter6/4.tex}
	
	\subsubsection*{\zihao{3} 6.5.\hspace{0.2cm}总结}
	\addcontentsline{toc}{subsubsection}{6.5.\hspace{0.2cm}总结}
	\subfile{content/2/chapter6/5.tex}
	
	\subsubsection*{\zihao{3} 6.6.\hspace{0.2cm}练习题}
	\addcontentsline{toc}{subsubsection}{6.6.\hspace{0.2cm}练习题}
	\subfile{content/2/chapter6/6.tex}
	\newpage
	
	\subsection*{\zihao{2} 第7章\hspace{0.5cm}集成代码质量工具}
	\addcontentsline{toc}{subsection}{第7章\hspace{0.5cm}集成代码质量工具}
	\subfile{content/2/chapter7/0.tex}
	
	\subsubsection*{\zihao{3} 7.1.\hspace{0.2cm}相关准备}
	\addcontentsline{toc}{subsubsection}{7.1.\hspace{0.2cm}相关准备}
	\subfile{content/2/chapter7/1.tex}
	
	\subsubsection*{\zihao{3} 7.2.\hspace{0.2cm}定义、发现和运行测试}
	\addcontentsline{toc}{subsubsection}{7.2.\hspace{0.2cm}定义、发现和运行测试}
	\subfile{content/2/chapter7/2.tex}
	
	\subsubsection*{\zihao{3} 7.3.\hspace{0.2cm}生成代码覆盖率报告}
	\addcontentsline{toc}{subsubsection}{7.3.\hspace{0.2cm}生成代码覆盖率报告}
	\subfile{content/2/chapter7/3.tex}
	
	\subsubsection*{\zihao{3} 7.4.\hspace{0.2cm}代码消杀}
	\addcontentsline{toc}{subsubsection}{7.4.\hspace{0.2cm}代码消杀}
	\subfile{content/2/chapter7/4.tex}
	
	\subsubsection*{\zihao{3} 7.5.\hspace{0.2cm}静态代码分析}
	\addcontentsline{toc}{subsubsection}{7.5.\hspace{0.2cm}静态代码分析}
	\subfile{content/2/chapter7/5.tex}
	
	\subsubsection*{\zihao{3} 7.6.\hspace{0.2cm}创建自定义构建类型}
	\addcontentsline{toc}{subsubsection}{7.6.\hspace{0.2cm}创建自定义构建类型}
	\subfile{content/2/chapter7/6.tex}
	
	\subsubsection*{\zihao{3} 7.7.\hspace{0.2cm}总结}
	\addcontentsline{toc}{subsubsection}{7.7.\hspace{0.2cm}总结}
	\subfile{content/2/chapter7/7.tex}
	
	\subsubsection*{\zihao{3} 7.8.\hspace{0.2cm}练习题}
	\addcontentsline{toc}{subsubsection}{7.8.\hspace{0.2cm}练习题}
	\subfile{content/2/chapter7/8.tex}
	\newpage
	
	\subsection*{\zihao{2} 第8章\hspace{0.5cm}执行自定义任务}
	\addcontentsline{toc}{subsection}{第8章\hspace{0.5cm}执行自定义任务}
	\subfile{content/2/chapter8/0.tex}
	
	\subsubsection*{\zihao{3} 8.1.\hspace{0.2cm}相关准备}
	\addcontentsline{toc}{subsubsection}{8.1.\hspace{0.2cm}相关准备}
	\subfile{content/2/chapter8/1.tex}
	
	\subsubsection*{\zihao{3} 8.2.\hspace{0.2cm}使用外部程序}
	\addcontentsline{toc}{subsubsection}{8.2.\hspace{0.2cm}使用外部程序}
	\subfile{content/2/chapter8/2.tex}
	
	\subsubsection*{\zihao{3} 8.3.\hspace{0.2cm}构建时执行自定义任务}
	\addcontentsline{toc}{subsubsection}{8.3.\hspace{0.2cm}构建时执行自定义任务}
	\subfile{content/2/chapter8/3.tex}
	
	\subsubsection*{\zihao{3} 8.4.\hspace{0.2cm}配置时执行自定义任务}
	\addcontentsline{toc}{subsubsection}{8.4.\hspace{0.2cm}配置时执行自定义任务}
	\subfile{content/2/chapter8/4.tex}
	
	\subsubsection*{\zihao{3} 8.5.\hspace{0.2cm}复制和修改文件}
	\addcontentsline{toc}{subsubsection}{8.5.\hspace{0.2cm}复制和修改文件}
	\subfile{content/2/chapter8/5.tex}
	
	\subsubsection*{\zihao{3} 8.6.\hspace{0.2cm}执行平台无关的命令}
	\addcontentsline{toc}{subsubsection}{8.6.\hspace{0.2cm}执行平台无关的命令}
	\subfile{content/2/chapter8/6.tex}
	
	\subsubsection*{\zihao{3} 8.7.\hspace{0.2cm}总结}
	\addcontentsline{toc}{subsubsection}{8.7.\hspace{0.2cm}总结}
	\subfile{content/2/chapter8/7.tex}
	
	\subsubsection*{\zihao{3} 8.8.\hspace{0.2cm}练习题}
	\addcontentsline{toc}{subsubsection}{8.8.\hspace{0.2cm}练习题}
	\subfile{content/2/chapter8/8.tex}
	\newpage
	
	\subsection*{\zihao{2} 第9章\hspace{0.5cm}创建可复制的构建环境}
	\addcontentsline{toc}{subsection}{第9章\hspace{0.5cm}创建可复制的构建环境}
	\subfile{content/2/chapter9/0.tex}
	
	\subsubsection*{\zihao{3} 9.1.\hspace{0.2cm}相关准备}
	\addcontentsline{toc}{subsubsection}{9.1.\hspace{0.2cm}相关准备}
	\subfile{content/2/chapter9/1.tex}
	
	\subsubsection*{\zihao{3} 9.2.\hspace{0.2cm}使用CMake预设}
	\addcontentsline{toc}{subsubsection}{9.2.\hspace{0.2cm}使用CMake预设}
	\subfile{content/2/chapter9/2.tex}
	
	\subsubsection*{\zihao{3} 9.3.\hspace{0.2cm}使用容器进行构建}
	\addcontentsline{toc}{subsubsection}{9.3.\hspace{0.2cm}使用容器进行构建}	
	\subfile{content/2/chapter9/3.tex}
	
	\subsubsection*{\zihao{3} 9.4.\hspace{0.2cm}使用sysroot隔离构建环境}
	\addcontentsline{toc}{subsubsection}{9.4.\hspace{0.2cm}使用sysroot隔离构建环境}
	\subfile{content/2/chapter9/4.tex}
	
	\subsubsection*{\zihao{3} 9.5.\hspace{0.2cm}总结}
	\addcontentsline{toc}{subsubsection}{9.5.\hspace{0.2cm}总结}
	\subfile{content/2/chapter9/5.tex}
	
	\subsubsection*{\zihao{3} 9.6.\hspace{0.2cm}练习题}
	\addcontentsline{toc}{subsubsection}{9.6.\hspace{0.2cm}练习题}
	\subfile{content/2/chapter9/6.tex}
	\newpage
	
	\subsection*{\zihao{2} 第10章\hspace{0.5cm}处理大项目和分布式存储库}
	\addcontentsline{toc}{subsection}{第10章\hspace{0.5cm}处理大项目和分布式存储库}
	\subfile{content/2/chapter10/0.tex}
	
	\subsubsection*{\zihao{3} 10.1.\hspace{0.2cm}相关准备}
	\addcontentsline{toc}{subsubsection}{10.1.\hspace{0.2cm}相关准备}
	\subfile{content/2/chapter10/1.tex}
	
	\subsubsection*{\zihao{3} 10.2.\hspace{0.2cm}超级构建的要求和需求}
	\addcontentsline{toc}{subsubsection}{10.2.\hspace{0.2cm}超级构建的要求和需求}
	\subfile{content/2/chapter10/2.tex}
	
	\subsubsection*{\zihao{3} 10.3.\hspace{0.2cm}跨多个代码库的构建}
	\addcontentsline{toc}{subsubsection}{10.3.\hspace{0.2cm}跨多个代码库的构建}
	\subfile{content/2/chapter10/3.tex}
	
	\subsubsection*{\zihao{3} 10.4.\hspace{0.2cm}超级构建中的版本一致性}
	\addcontentsline{toc}{subsubsection}{10.4.\hspace{0.2cm}超级构建中的版本一致性}
	\subfile{content/2/chapter10/4.tex}
	
	\subsubsection*{\zihao{3} 10.5.\hspace{0.2cm}总结}
	\addcontentsline{toc}{subsubsection}{10.5.\hspace{0.2cm}总结}
	\subfile{content/2/chapter10/5.tex}
	
	\subsubsection*{\zihao{3} 10.6.\hspace{0.2cm}练习题}
	\addcontentsline{toc}{subsubsection}{10.6.\hspace{0.2cm}练习题}
	\subfile{content/2/chapter10/6.tex}
	\newpage
	
	\subsection*{\zihao{2} 第11章\hspace{0.5cm}自动化模糊测试}
	\addcontentsline{toc}{subsection}{第11章\hspace{0.5cm}自动化模糊测试}
	\subfile{content/2/chapter11/0.tex}
	
	\subsubsection*{\zihao{3} 11.1.\hspace{0.2cm}相关准备}
	\addcontentsline{toc}{subsubsection}{11.1.\hspace{0.2cm}相关准备}
	\subfile{content/2/chapter11/1.tex}
	
	\subsubsection*{\zihao{3} 11.2.\hspace{0.2cm}简介模糊测试}
	\addcontentsline{toc}{subsubsection}{11.2.\hspace{0.2cm}简介模糊测试}
	\subfile{content/2/chapter11/2.tex}
	
	\subsubsection*{\zihao{3} 11.3.\hspace{0.2cm}集成AFL/libfuzzer}
	\addcontentsline{toc}{subsubsection}{11.3.\hspace{0.2cm}集成AFL/libfuzzer}
	\subfile{content/2/chapter11/3.tex}
	
	\subsubsection*{\zihao{3} 11.4.\hspace{0.2cm}总结}
	\addcontentsline{toc}{subsubsection}{11.4.\hspace{0.2cm}总结}
	\subfile{content/2/chapter11/4.tex}
	
	\subsubsection*{\zihao{3} 11.5.\hspace{0.2cm}练习题}
	\addcontentsline{toc}{subsubsection}{11.5.\hspace{0.2cm}练习题}
	\subfile{content/2/chapter11/5.tex}
	\newpage
	
	\color{white}
	\section*{\zihao{1}第三部分:掌控细节}
	\pagecolor{mygray}
	\addcontentsline{toc}{section}{第三部分:掌控细节}
	\textbf{\subfile{content/3/Section.tex}}
	\newpage
	\color{black}
	\pagecolor{white}
	\newpage
	
	\subsection*{\zihao{2} 第12章\hspace{0.5cm}跨平台编译和自定义工具链}
	\addcontentsline{toc}{subsection}{第12章\hspace{0.5cm}跨平台编译和自定义工具链}
	\subfile{content/3/chapter12/0.tex}
	
	\subsubsection*{\zihao{3} 12.1.\hspace{0.2cm}相关准备}
	\addcontentsline{toc}{subsubsection}{12.1.\hspace{0.2cm}相关准备}
	\subfile{content/3/chapter12/1.tex}
	
	\subsubsection*{\zihao{3} 12.2.\hspace{0.2cm}使用跨平台工具链}
	\addcontentsline{toc}{subsubsection}{12.2.\hspace{0.2cm}使用跨平台工具链}
	\subfile{content/3/chapter12/2.tex}
	
	\subsubsection*{\zihao{3} 12.3.\hspace{0.2cm}创建工具链}
	\addcontentsline{toc}{subsubsection}{12.3.\hspace{0.2cm}创建工具链}
	\subfile{content/3/chapter12/3.tex}
	
	\subsubsection*{\zihao{3} 12.4.\hspace{0.2cm}测试交叉编译的二进制文件}
	\addcontentsline{toc}{subsubsection}{12.4.\hspace{0.2cm}测试交叉编译的二进制文件}
	\subfile{content/3/chapter12/4.tex}
	
	\subsubsection*{\zihao{3} 12.5.\hspace{0.2cm}总结}
	\addcontentsline{toc}{subsubsection}{12.5.\hspace{0.2cm}总结}
	\subfile{content/3/chapter12/5.tex}
	
	\subsubsection*{\zihao{3} 12.6.\hspace{0.2cm}练习题}
	\addcontentsline{toc}{subsubsection}{12.6.\hspace{0.2cm}练习题}
	\subfile{content/3/chapter12/6.tex}
	\newpage
	
	\subsection*{\zihao{2} 第13章\hspace{0.5cm}重用CMake代码}
	\addcontentsline{toc}{subsection}{第13章\hspace{0.5cm}重用CMake代码}
	\subfile{content/3/chapter13/0.tex}
	
	\subsubsection*{\zihao{3} 13.1.\hspace{0.2cm}相关准备}
	\addcontentsline{toc}{subsubsection}{13.1.\hspace{0.2cm}相关准备}
	\subfile{content/3/chapter13/1.tex}
	
	\subsubsection*{\zihao{3} 13.2.\hspace{0.2cm}了解CMake模块}
	\addcontentsline{toc}{subsubsection}{13.2.\hspace{0.2cm}了解CMake模块}
	\subfile{content/3/chapter13/2.tex}
	
	\subsubsection*{\zihao{3} 13.3.\hspace{0.2cm}模块的基本构建块——函数和宏}
	\addcontentsline{toc}{subsubsection}{13.3.\hspace{0.2cm}模块的基本构建块——函数和宏}
	\subfile{content/3/chapter13/3.tex}
	
	\subsubsection*{\zihao{3} 13.4.\hspace{0.2cm}编写一个CMake模块}
	\addcontentsline{toc}{subsubsection}{13.4.\hspace{0.2cm}编写一个CMake模块}
	\subfile{content/3/chapter13/4.tex}
	
	\subsubsection*{\zihao{3} 13.5.\hspace{0.2cm}总结}
	\addcontentsline{toc}{subsubsection}{13.5.\hspace{0.2cm}总结}
	\subfile{content/3/chapter13/5.tex}
	
	\subsubsection*{\zihao{3} 13.6.\hspace{0.2cm}练习题}
	\addcontentsline{toc}{subsubsection}{13.6.\hspace{0.2cm}练习题}
	\subfile{content/3/chapter13/6.tex}
	\newpage
	
	\subsection*{\zihao{2} 第14章\hspace{0.5cm}优化和维护CMake项目}
	\addcontentsline{toc}{subsection}{第14章\hspace{0.5cm}优化和维护CMake项目}
	\subfile{content/3/chapter14/0.tex}
	
	\subsubsection*{\zihao{3} 14.1.\hspace{0.2cm}相关准备}
	\addcontentsline{toc}{subsubsection}{14.1.\hspace{0.2cm}相关准备}
	\subfile{content/3/chapter14/1.tex}
	
	\subsubsection*{\zihao{3} 14.2.\hspace{0.2cm}保持CMake项目的可维护性}
	\addcontentsline{toc}{subsubsection}{14.2.\hspace{0.2cm}保持CMake项目的可维护性}
	\subfile{content/3/chapter14/2.tex}
	
	\subsubsection*{\zihao{3} 14.3.\hspace{0.2cm}对CMake构建进行性能分析}
	\addcontentsline{toc}{subsubsection}{14.3.\hspace{0.2cm}对CMake构建进行性能分析}
	\subfile{content/3/chapter14/3.tex}
	
	\subsubsection*{\zihao{3} 14.4.\hspace{0.2cm}优化构建性能}
	\addcontentsline{toc}{subsubsection}{14.4.\hspace{0.2cm}优化构建性能}
	\subfile{content/3/chapter14/4.tex}
	
	\subsubsection*{\zihao{3} 14.5.\hspace{0.2cm}总结}
	\addcontentsline{toc}{subsubsection}{14.5.\hspace{0.2cm}总结}
	\subfile{content/3/chapter14/5.tex}
	
	\subsubsection*{\zihao{3} 14.6.\hspace{0.2cm}练习题}
	\addcontentsline{toc}{subsubsection}{14.6.\hspace{0.2cm}练习题}
	\subfile{content/3/chapter14/6.tex}
	\newpage
	
	\subsection*{\zihao{2} 第15章\hspace{0.5cm}迁移到CMake}
	\addcontentsline{toc}{subsection}{第15章\hspace{0.5cm}迁移到CMake}
	\subfile{content/3/chapter15/0.tex}
	
	\subsubsection*{\zihao{3} 15.1.\hspace{0.2cm}相关准备}
	\addcontentsline{toc}{subsubsection}{15.1.\hspace{0.2cm}相关准备}
	\subfile{content/3/chapter15/1.tex}
	
	\subsubsection*{\zihao{3} 15.2.\hspace{0.2cm}高级迁移策略}
	\addcontentsline{toc}{subsubsection}{15.2.\hspace{0.2cm}高级迁移策略}
	\subfile{content/3/chapter15/2.tex}
	
	\subsubsection*{\zihao{3} 15.3.\hspace{0.2cm}迁移小项目}
	\addcontentsline{toc}{subsubsection}{15.3.\hspace{0.2cm}迁移小项目}
	\subfile{content/3/chapter15/3.tex}
	
	\subsubsection*{\zihao{3} 15.4.\hspace{0.2cm}将大型项目迁移到CMake}
	\addcontentsline{toc}{subsubsection}{15.4.\hspace{0.2cm}将大型项目迁移到CMake}
	\subfile{content/3/chapter15/4.tex}
	
	\subsubsection*{\zihao{3} 15.5.\hspace{0.2cm}总结}
	\addcontentsline{toc}{subsubsection}{15.5.\hspace{0.2cm}总结}
	\subfile{content/3/chapter15/5.tex}
	
	\subsubsection*{\zihao{3} 15.6.\hspace{0.2cm}练习题}
	\addcontentsline{toc}{subsubsection}{15.6.\hspace{0.2cm}练习题}
	\subfile{content/3/chapter15/6.tex}
	\newpage
	
	\subsection*{\zihao{2} 第16章\hspace{0.5cm}对CMake进行贡献}
	\addcontentsline{toc}{subsection}{第16章\hspace{0.5cm}对CMake进行贡献}
	\subfile{content/3/chapter16/0.tex}
	
	\subsubsection*{\zihao{3} 16.1.\hspace{0.2cm}预备知识}
	\addcontentsline{toc}{subsubsection}{16.1.\hspace{0.2cm}预备知识}
	\subfile{content/3/chapter16/1.tex}
	
	\subsubsection*{\zihao{3} 16.2.\hspace{0.2cm}找到CMake社区}
	\addcontentsline{toc}{subsubsection}{16.2.\hspace{0.2cm}找到CMake社区}
	\subfile{content/3/chapter16/2.tex}
	
	\subsubsection*{\zihao{3} 16.3.\hspace{0.2cm}为CMake贡献力量}
	\addcontentsline{toc}{subsubsection}{16.3.\hspace{0.2cm}为CMake贡献力量}
	\subfile{content/3/chapter16/3.tex}
	
	\subsubsection*{\zihao{3} 16.4.\hspace{0.2cm}推荐的书籍和博客}
	\addcontentsline{toc}{subsubsection}{16.4.\hspace{0.2cm}推荐的书籍和博客}
	\subfile{content/3/chapter16/4.tex}
	
	\subsubsection*{\zihao{3} 16.5.\hspace{0.2cm}总结}
	\addcontentsline{toc}{subsubsection}{16.5.\hspace{0.2cm}总结}
	\subfile{content/3/chapter16/5.tex}
	\newpage
	
	\section*{\zihao{2} 参考答案}
	\addcontentsline{toc}{section}{参考答案}
	\subfile{content/Assessments/0.tex}
	
	\subsubsection*{\zihao{3} 第1章}
	\addcontentsline{toc}{subsubsection}{第1章}
	\subfile{content/Assessments/1.tex}
	
	\subsubsection*{\zihao{3} 第2章}
	\addcontentsline{toc}{subsubsection}{第2章}
	\subfile{content/Assessments/2.tex}
	
	\subsubsection*{\zihao{3} 第3章}
	\addcontentsline{toc}{subsubsection}{第3章}
	\subfile{content/Assessments/3.tex}
	
	\subsubsection*{\zihao{3} 第4章}
	\addcontentsline{toc}{subsubsection}{第4章}
	\subfile{content/Assessments/4.tex}
	
	\subsubsection*{\zihao{3} 第5章}
	\addcontentsline{toc}{subsubsection}{第5章}
	\subfile{content/Assessments/5.tex}
	
	\subsubsection*{\zihao{3} 第6章}
	\addcontentsline{toc}{subsubsection}{第6章}
	\subfile{content/Assessments/6.tex}
	
	\subsubsection*{\zihao{3} 第7章}
	\addcontentsline{toc}{subsubsection}{第7章}
	\subfile{content/Assessments/7.tex}
	
	\subsubsection*{\zihao{3} 第8章}
	\addcontentsline{toc}{subsubsection}{第8章}
	\subfile{content/Assessments/8.tex}
	
	\subsubsection*{\zihao{3} 第9章}
	\addcontentsline{toc}{subsubsection}{第9章}
	\subfile{content/Assessments/9.tex}
	
	\subsubsection*{\zihao{3} 第10章}
	\addcontentsline{toc}{subsubsection}{第10章}
	\subfile{content/Assessments/10.tex}
	
	\subsubsection*{\zihao{3} 第11章}
	\addcontentsline{toc}{subsubsection}{第11章}
	\subfile{content/Assessments/11.tex}
	
	\subsubsection*{\zihao{3} 第12章}
	\addcontentsline{toc}{subsubsection}{第12章}
	\subfile{content/Assessments/12.tex}
	
	\subsubsection*{\zihao{3} 第13章}
	\addcontentsline{toc}{subsubsection}{第13章}
	\subfile{content/Assessments/13.tex}
	
	\subsubsection*{\zihao{3} 第14章}
	\addcontentsline{toc}{subsubsection}{第14章}
	\subfile{content/Assessments/14.tex}
	
	\subsubsection*{\zihao{3} 第15章}
	\addcontentsline{toc}{subsubsection}{第15章}
	\subfile{content/Assessments/15.tex}
	\newpage

\end{sloppypar}
\end{document}

