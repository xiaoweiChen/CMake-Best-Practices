

\begin{enumerate}
\item 
Toolchain files are passed either by the --toolchain command-line flag, the CMAKE\_TOOLCHAIN\_FILE variable, or with the toolchainFile option in a CMake preset.

\item
Usually, the following things are done in a toolchain file for cross-compiling:

\begin{enumerate}[label=\Alph*]
\item
Defining the target system and architecture
 
\item 
Providing paths to any tools needed to build the software

\item 
Setting default flags for the compiler and linkers

\item 
Pointing to the sysroot and possibly any staging directory if cross-compiling

\item 
Setting hints for the search order for any find\_ commands of CMake
\end{enumerate}

\item 
The staging directory is set with the CMAKE\_STAGING\_PREFIX variable and is used as a place to install any built artifacts if the sysroot should not be modified.

\item 
The emulator command is passed as a semicolon-separated list in the CMAKE\_CROSSCOMPILING\_EMULATOR variable.

\item 
Any call to project() or enable\_language() in a project will trigger
detection of the features.

\item 
The configuration context for compiler checks can be stored with cmake\_push\_check\_state() and restored to a previous state with cmake\_pop\_check\_state().

\item 
If CMAKE\_CROSSCOMPILING is set, any call to try\_run() will only compile the test but not run it unless an emulator command is set.

\item 
Build directories should be fully cleared because the temporary artifacts for compiler checks might not be rebuilt properly when just deleting the cache.
\end{enumerate}