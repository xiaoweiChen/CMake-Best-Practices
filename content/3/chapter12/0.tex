CMake支持软件的跨平台构建,只要运行CMake的系统上有必要的工具,就可以为其他平台构建项目。构建软件时,通常要了解编译器和链接器,这些是构建软件的基本工具,但在构建软件时还有一些其他工具、库和文件,这些统称为工具链。

本书所有的示例都是运行在同一系统上构建,CMake通常能很好地找到正确的工具链。然而,若软件是为另一个平台构建,则工具链通常由开发人员指定。工具链的定义可能相对简单,只指定目标平台,也可能复杂到指定构建软件所需的每个工具的路径,或者为特定芯片组创建二进制文件所需的特定编译器标志。

交叉编译中,工具链通常和系统根目录(sysroot)一起使用。sysroot是目标平台的精简文件系统目录。构建软件时,将其视为根文件夹,用于查找必要的库和文件,以便为预期的目标平台进行编译和链接。

虽然交叉编译开始可能令人生畏,但当正确使用CMake后,就没那么困难了。本章将介绍如何使用工具链文件,以及如何自己编写工具链文件。我们将详细研究在构建软件的特定阶段使用了哪些工具。最后,再来了解如何设置CMake,从而可以使用模拟器运行测试。

我们将讨论以下主题:

\begin{itemize}
\item 
使用跨平台工具链

\item 
创建工具链

\item 
测试交叉编译的二进制文件

\item 
测试工具链支持的功能
\end{itemize}

本章结束时,将能够处理现有的工具链,以及如何使用CMake为不同平台构建和测试软件。我们将深入研究如何测试编译器的某个特性,以确定是否符合我们的预期。
















