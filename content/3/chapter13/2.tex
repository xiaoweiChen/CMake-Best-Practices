CMake模块包含CMake代码、函数和宏,放在一起以服务于特定的目的。模块可以为其他CMake代码提供函数和宏,并在包含时执行CMake命令。通常,CMake内置了许多预先制作好的模块。这些模块提供了实用程序来使用CMake代码,并允许发现第三方工具和依赖项(Find*.cmake模块)。CMake默认提供的模块列表可在\url{https://cmake.org/cmake/help/latest/manual/cmakemodules.7.html}获得。官方文档将模块分为以下两类:

\begin{itemize}
\item 
实用工具模块

\item 
查找模块
\end{itemize}

实用工具模块提供相应的工具,而查找模块用于搜索系统中的第三方软件。我们已经在第4章和第5章中详细介绍了查找模块。因此,在本章中专门关注实用工具模块。前面的章节中使用了一些CMake提供的实用模块,其中一些模块是GNUInstallDirs、CPack、FetchContent和ExternalProject,这些模块位于CMake安装文件夹下。

为了更好地理解实用工具模块的概念,先从研究CMake提供的实用程序模块开始,先来看看ProcessorCount实用程序模块。可以在\url{https://github.com/Kitware/CMake/blob/master/Modules/ProcessorCount.cmake}找到这个模块的源文件。ProcessorCount模块可在CMake代码中检索系统的CPU核心计数的模块。ProcessorCount.cmake定义了名为ProcessorCount的函数,其接受名为var的参数。该函数的实现大致如下:

\begin{lstlisting}[style=styleCMake]
function(ProcessorCount var)
	# Unknown:
	set(count 0)
	if(WIN32)
		set(count "$ENV{NUMBER_OF_PROCESSORS}")
	endif()
	if(NOT count)
		# Mac, FreeBSD, OpenBSD (systems with sysctl):
		# … mac-specific approach … #
	endif()
	if(NOT count)
		# Linux (systems with nproc):
		# … linux-specific approach … #
	endif()
# … Other platforms, alternative fallback methods … #
# Lastly:
set(${var} ${count} PARENT_SCOPE)
endfunction()
\end{lstlisting}

ProcessorCount函数尝试几种不同的方法来检索主机的CPU核心计数。ProcessorCount模块的用法很简单:

\begin{lstlisting}[style=styleCMake]
include(ProcessorCount)
ProcessorCount(CORE_COUNT)
message(STATUS "Core count: ${CORE_COUNT}")
\end{lstlisting}

使用CMake模块非常简单,只需将该模块包含到所需的CMake文件中即可。\texttt{include()}具有传递性,此行之后的代码可以使用模块中包含的所有CMake定义。

现在我们对实用工具模块有了大致的了解。继续了解更多关于实用程序模块的基本构建模块:函数和宏。



























