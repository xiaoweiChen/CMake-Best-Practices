There are a lot of books, blogs, and resources regarding CMake. The following is a curated list of hand-selected resources that you may find useful. This list will allow you to discover more about CMake and expand your horizons:

\begin{itemize}
\item 
CMake official documentation: \url{https://cmake.org/documentation/}

The official documentation for CMake. It is very broad and up to date.
	
\item 
awesome-cmake: \url{https://github.com/onqtam/awesome-cmake}

A vast curation of CMake-related resources. It is very extensive and regularly updated.

\item 
Getting Started with CMake: Helpful Resources: \url{https://embeddedartistry.com/blog/2017/10/04/getting-started-with-cmake-helpfulresources/}

A curation of helpful CMake resources gathered together by Embedded Artistry

\item 
An Introduction to Modern CMake: \url{https://cliutils.gitlab.io/modern-cmake/}

An online book and a good resource about learning modern CMake in detail. It is
written by Henry Schreiner and many other contributors.

\item 
More Modern CMake: \url{https://hsf-training.github.io/hsf-training-cmake-webpage/01-intro/index.html}

This is a follow-up book to An Introduction to Modern CMake, written by the HEP Software Foundation.

\item 
More Modern CMake: \url{https://www.youtube.com/watch?v=y7ndUhdQuU8}

This YouTube video is a presentation performed by Deniz Bahadir at Meeting C++ 2018. Its main aim is to give tips about using CMake correctly.

\item 
Deep CMake for Library Authors: \url{https://www.youtube.com/watch?v=m0DwB4OvDXk}

This YouTube video is a CppCon talk given by Craig Scott, co-maintainer of the CMake project. It covers CMake topics oriented toward library development.

\item 
Daniel Pfeifer's Effective CMake: \url{https://www.youtube.com/watch?v=bsXLMQ6WgIk}

This YouTube video is a talk given by Daniel Pfeifer about using CMake effectively. It covers overall CMake usage.

\item 
Professional CMake: A Practical Guide: \url{https://crascit.com/professional-cmake/}

A comprehensive book written by CMake's co-maintainer, Craig Scott. It is very extensive and contains many details that you can't find elsewhere.

\item 
learning-cmake: \url{https://github.com/Akagi201/learning-cmake}

This repository has a collection of examples for the purpose of learning different usecases in CMake.

\item 
cmake-examples: \url{https://github.com/ttroy50/cmake-examples}

Another good collection of examples for the purpose of learning different use cases in CMake.
\end{itemize}

With that said, we have reached another chapter's end. Next, we'll summarize what we have learned in this chapter.
























