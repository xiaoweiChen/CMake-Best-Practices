
After getting involved with CMake, you may feel the need to exchange ideas and find a platform for asking questions of people that will likely know the answers. For that purpose, I have a few online platform recommendations for you.

\subsubsubsection{16.2.1\hspace{0.2cm}Stack Overflow}

Stack Overflow is a popular Q\&A platform and the go-to place for most developers. If you are having an issue with CMake or have any questions, you can search for answers to your questions on Stack Overflow first. It is highly likely that someone has experienced that issue or asked the same or similar questions before. You can also take a look at the popular questions list to discover new ways of doing things with CMake.

When asking questions, ensure that you are labeling your questions with the cmake tag. This will allow individuals who are interested in answering CMake-related questions to find your question easier. You can access the Stack Overflow home page at \url{https://stackoverflow.com/}.

\subsubsubsection{16.2.2\hspace{0.2cm}Reddit (r/cmake)}

Reddit is a popular place that has dedicated, separate bulletin-like areas named subreddits to exchange ideas around topics. Reddit also has an r/cmake subreddit that contains CMake-specific questions, announcements, and shares. You can discover many useful GitHub repositories, get notified about recent CMake releases, and discover blog posts and materials that can help you. You can access the r/cmake subreddit at \url{https://www.reddit.com/r/cmake/}.

\subsubsubsection{16.2.3\hspace{0.2cm}The CMake Discourse forum}

The CMake Discourse forum is the main place for CMake developers and users to meet. It is completely dedicated to CMake-specific matters only. The forum contains  announcements, guides on how to use CMake, a community space, a CMake development space, and much more content that you may be interested in. You can access the discourse forum at \url{https://discourse.cmake.org/}.

\subsubsubsection{16.2.4\hspace{0.2cm}The Kitware CMake GitLab repository}

Kitware's CMake repository is also a good resource for finding solutions for issues you may experience. Try searching for the issue you have in the issue list available at \url{https://gitlab.kitware.com/cmake/cmake/-/issues}. There is a good chance that somebody else may already have filed an issue about the topic. If that's not the case, you can create a new issue by adhering to CMake's contributing rules.

The preceding list is non-exhaustive and many more forums are available online. These four platforms will be sufficient to get you started. Next, we will talk about ways of contributing to the CMake project itself.