Documentation is—without a doubt—an essential part of all projects. Documentation conveys information that is not implicitly available to the user. It is a way of sharing the intent, functionality, capabilities, and restrictions regarding a project and it enables both technical and non-technical people to work on a project. But it is indeed a time-consuming process to write documentation, thus it is crucial to make use of the tools available for generating documentation.

This chapter will look into ways of integrating Doxygen, DOT, and PlantUML into CMake to speed up the documentation process. These tools will allow us to lessen the context switch between code and documentation and also ease the maintenance burden of documentation.

To understand the skills shared in this chapter, we'll cover the following main topics:

\begin{itemize}
\item 
Generating documentation from your code

\item 
Packaging and distributing documentation with CPack

\item 
Creating dependency graphs of CMake targets
\end{itemize}

Let's begin with the technical requirements.







