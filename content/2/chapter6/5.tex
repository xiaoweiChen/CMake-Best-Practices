In this chapter, we briefly introduced Doxygen and learned how to generate documentation from code, as well as how to package generated documentation for deployment. It is crucial to have these skills under your belt when it comes to any software project. Generating documentation from code greatly reduces the effort for technical documentation and has virtually no maintenance cost. From the perspective of a software professional, automating deterministic stuff and generating inferable information in different representations is most desired. This approach creates space and time for other engineering tasks that require more human problem-solving skills. Automating tasks reduces the maintenance cost, makes the product more stable, and reduces the overall need for human resources. It is a way of converting pure human effort to spent electricity by enabling a machine to do the same job. Machines are amazingly better at performing deterministic jobs than humans. They are never sick, rarely broken, and are easily scaled and never tired. Automation is a way of harnessing this untamed power.

The main goal of this book is not to teach you how to do things, but to teach you how to make a machine work for a particular task. This approach indeed requires learning first, but keep in mind that if you are doing a costly operation that can be done by a machine manually more than once, you are wasting your precious time. Invest in automating things—it is a profitable investment that pays for itself quickly.

In the next chapter, we will be learning how we can improve our code quality by integrating unit testing, code sanitizers, static code analysis, micro-benchmarking, and code coverage tools into our CMake projects, and of course, we will be automating all of this as well.

See you in the next chapter!