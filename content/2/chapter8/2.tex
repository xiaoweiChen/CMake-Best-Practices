CMake has a pretty broad functionality so that it can cover many tasks when building software. However, there are situations when developers will need to do something that is not covered. Common examples include running special tools that do some pre-or postprocessing of files for a target, using source code generators that produce input for the compiler, and compressing and archiving artifacts that are not handled with CPack. The list of such special tasks that must be accomplished during a build step is probably almost endless. CMake supports three ways of executing custom tasks:

\begin{itemize}
\item 
By defining a target that executes a command with add\_custom\_target

\item 
By attaching a custom command to an existing target by using add\_custom\_command or by making a target depend on a file that's been generated by a custom command

\item 
By using the execute\_process function, which executes a command during the configuration step
\end{itemize}

Whenever possible, external programs should be called during the build step because the configuration step is far less controllable by the user and should generally run as fast as possible.

Let's learn how to define tasks that run at build time.